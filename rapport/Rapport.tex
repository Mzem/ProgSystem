\documentclass[a4]{article}
\usepackage[utf8]{inputenc}
\usepackage[french]{babel}
\usepackage{listings}
\usepackage{color}

\definecolor{mygreen}{rgb}{0,0.6,0}
\definecolor{mygray}{rgb}{0.5,0.5,0.5}
\definecolor{mymauve}{rgb}{0.58,0,0.82}

\lstset{
  backgroundcolor=\color{white},   % choose the background color; you must add \usepackage{color} or \usepackage{xcolor}
  basicstyle=\footnotesize,        % the size of the fonts that are used for the code
  breakatwhitespace=false,         % sets if automatic breaks should only happen at whitespace
  breaklines=true,                 % sets automatic line breaking
  captionpos=b,                    % sets the caption-position to bottom
  commentstyle=\color{mygreen},    % comment style
  deletekeywords={...},            % if you want to delete keywords from the given language
  escapeinside={\%*}{*)},          % if you want to add LaTeX within your code
  extendedchars=true,              % lets you use non-ASCII characters; for 8-bits encodings only, does not work with UTF-8
  frame=L,	                       % adds a frame around the code
  keepspaces=true,                 % keeps spaces in text, useful for keeping indentation of code (possibly needs columns=flexible)
  keywordstyle=\color{blue},       % keyword style
  language=C,                 	   % the language of the code
  otherkeywords={*,...},           % if you want to add more keywords to the set
  numbers=none,                    % where to put the line-numbers; possible values are (none, left, right)
  numbersep=5pt,                   % how far the line-numbers are from the code
  numberstyle=\tiny\color{mygray}, % the style that is used for the line-numbers
  rulecolor=\color{black},         % if not set, the frame-color may be changed on line-breaks within not-black text (e.g. comments (green here))
  showspaces=false,                % show spaces everywhere adding particular underscores; it overrides 'showstringspaces'
  showstringspaces=false,          % underline spaces within strings only
  showtabs=false,                  % show tabs within strings adding particular underscores
  stepnumber=2,                    % the step between two line-numbers. If it's 1, each line will be numbered
  stringstyle=\color{mymauve},     % string literal style
  tabsize=2,	                   % sets default tabsize to 2 spaces
  title=\lstname                   % show the filename of files included with \lstinputlisting; also try caption= instead of title
}
%gestion des caractères latins (accents...)
\lstset{literate=
  {á}{{\'a}}1 {é}{{\'e}}1 {í}{{\'i}}1 {ó}{{\'o}}1 {ú}{{\'u}}1
  {Á}{{\'A}}1 {É}{{\'E}}1 {Í}{{\'I}}1 {Ó}{{\'O}}1 {Ú}{{\'U}}1
  {à}{{\`a}}1 {è}{{\`e}}1 {ì}{{\`i}}1 {ò}{{\`o}}1 {ù}{{\`u}}1
  {À}{{\`A}}1 {È}{{\'E}}1 {Ì}{{\`I}}1 {Ò}{{\`O}}1 {Ù}{{\`U}}1
  {ä}{{\"a}}1 {ë}{{\"e}}1 {ï}{{\"i}}1 {ö}{{\"o}}1 {ü}{{\"u}}1
  {Ä}{{\"A}}1 {Ë}{{\"E}}1 {Ï}{{\"I}}1 {Ö}{{\"O}}1 {Ü}{{\"U}}1
  {â}{{\^a}}1 {ê}{{\^e}}1 {î}{{\^i}}1 {ô}{{\^o}}1 {û}{{\^u}}1
  {Â}{{\^A}}1 {Ê}{{\^E}}1 {Î}{{\^I}}1 {Ô}{{\^O}}1 {Û}{{\^U}}1
  {œ}{{\oe}}1 {Œ}{{\OE}}1 {æ}{{\ae}}1 {Æ}{{\AE}}1 {ß}{{\ss}}1
  {ű}{{\H{u}}}1 {Ű}{{\H{U}}}1 {ő}{{\H{o}}}1 {Ő}{{\H{O}}}1
  {ç}{{\c c}}1 {Ç}{{\c C}}1 {ø}{{\o}}1 {å}{{\r a}}1 {Å}{{\r A}}1
  {€}{{\EUR}}1 {£}{{\pounds}}1
}

\author{Sonny Klotz - Younes Ben Yamna - Malek Zemni}
\title{Rapport - Projet systèmes d'exploitation}
\date{\today}

\begin{document}
\maketitle

	\section{Introduction}
			Le but de ce programme est d'effectuer une opération donnée \textit{(minimum, maximum, moyenne, somme ou impairs)} sur
			un ensemble de valeurs répartis dans plusieurs fichiers.\\
			Dans un premier lieu, cette opération sera effectuée localement sur chaque fichier. Dans un second lieu, cette opération sera effectuée
			sur les résultats des opérations locales aux fichiers pour obtenir un résultat final.\\
			Cette manipulation va donc nécessiter une réduction à deux niveaux : le travail sera d'abord divisé entre N processus, puis entre M 
			threads pas processus.\\
			La réalisation de cette opération de réduction va donc nécessiter l'implémentation de deux fonctions principales :
			\begin{itemize}
				\item{La fonction \textbf{directeur} : effectue la première réduction}
				\item{La fonction \textbf{chef} : effectue la seconde réduction}
			\end{itemize}
			La fonction \textbf{directeur} va divise le travail entre N processus fils. Chaque processus fils quant à lui va faire appel à la fonction 
			\textbf{chef} qui divise le travail entre M threads.
			
	\section{Directeur}
			La fonction \textbf{directeur} et toutes ses fonctionnalités sont codées dans le fichier \textbf{directeur.c}.\\
			Le role principal de cette fonction est de créer autant de processus que de fichiers fournis en utilisant l'appel sysème 
			\textbf{fork()}.\\
			Dans la fonction \textbf{directeur}, une première boucle permet d'abord de créer tous les processus et de lancer la fonction 
			\textbf{chef} pour chaque processus. Ensuite, une deuxième boucle permet au processus père d'attendre la fin de tous ses processus fils.\\
			La gestion d'erreurs de la fonction \textbf{fork()} est assurée par la variable globale \textit{errno} fournie par errno.h.\\
			
			La récupération du résultat...?\\
			
	\section{Chef}
			\paragraph{Structure inf :\\}
			Strucutre utilisée pour...
			\lstinputlisting[firstline=26,lastline=33,title=donnees.h]{../src/donnees.h}
			
	\section{Fonctions}

	\section{Conclusion}
		

\end{document}


