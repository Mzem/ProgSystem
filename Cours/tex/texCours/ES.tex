\section{Entrées / Sorties}
	Les entrées / sorties traitent des données qu'on peut lire ou écrire à partir de fichiers. Il existe 2 niveaux de gestion des entrées / sorties et fichiers :
	\begin{itemize}
		\item Entrées / sorties effectuées immédiatement (avec des appels-systèmes)
		\item Entrées / sorties où les données sont mises en mémoire temporaire (\textit{\textbf{buffer} ou mémoire tampon, zone de mémoire virtuelle utilisée pour stocker temporairement les données, notamment entre 2 processus}) (avec des fonctions standard)
	\end{itemize}
	
	\subsection{Entrées / Sorties standard}
		Toutes les fonctions décrites ici sont déclarées dans la bibliothèque \lstinline!stdio.h!.
		
		~\\\textbf{Ouvrir et fermer un fichier :}
			
			\begin{lstlisting}
				FILE* fopen(const char* nomDuFichier, const char* modeOuverture);
				// renvoie NULL si l'ouverture échoue
				
				int fclose(FILE* pointeurSurFichier);
				// renvoie 0 si la fermeture réussit, EOF sinon
			\end{lstlisting}
		
		~\\\textbf{Écrire dans un fichier :}
		
			\begin{lstlisting}
				int fputc(int caractere, FILE* pointeurSurFichier);
				// écrit un seul caractère à la fois dans le fichier
				
				char* fputs(const char* chaine, FILE* pointeurSurFichier);
				// écrit une chaîne dans le fichier
				
				int fprintf(FILE* pointeurDeFichier, const char *format, ...); 
				// écrit une chaîne formatée dans le fichier, fonctionnement quasi-identique à printf
				
				// retournent EOF si écriture échoue
			\end{lstlisting}
			
		~\\\textbf{Lire dans un fichier :}
		
			\begin{lstlisting}
				int fgetc(FILE* pointeurDeFichier);
				// lit un caractère, et avance la tête de lecture
				// retourne le caractère lu ou EOF sinon
				
				char* fgets(char* chaine, int nbreDeCaracteresALire, FILE* pointeurSurFichier);
				// lit au maximum une ligne et s'arrête au premier \lstinline!\n!
				// <nbreDeCaracteresALire> : pour s'arrêter le lire avant la fin de ligne, sert à définir une taille max
				// retourne NULL si elle ne peut rien lire
				
				int fscanf(FILE* pointeurDeFichier, const char *format, ...);
				// écrit une chaîne formatée
			\end{lstlisting}
			
		~\\\textbf{Se déplacer dans un fichier :} la tête de lecture est une sorte de curseur virtuel dans un fichier qui indique la position de lecture / écriture actuelle.
		
			\begin{lstlisting}
				long ftell(FILE* pointeurSurFichier);
				// indique la position actuelle dans le fichier
				
				int fseek(FILE* pointeurSurFichier, long deplacement, int origine);
				// déplace le curseur de <deplacement> caractères à partir de la position <origine>
				// <deplacement> peut être positif (en avant), 0 ou négatif (en arrière)
				// <origine> peut prendre les constantes SEEK_SET (début), SEEK_CUR (actuelle) ou SEEK_END (fin)
				
				void rewind(FILE* pointeurSurFichier);
				// retour au début
			\end{lstlisting}
			
		
	\subsection{Entrées / Sorties système}
		Les fonctions d'entrée / sortie système sont des appels-systèmes communiquant directement avec le noyau. Les fonctions d'entrée / sortie standard font eux-même appel à ces fonctions système. Ces fonctions système se trouvent dans les bibliothèques suivantes :
		\begin{lstlisting}
			#include <sys/types.h>
			#include <sys/stat.h>
			#include <fcntl.h>
			#include <unistd.h>
		\end{lstlisting}
		
		
		~\\\textbf{Ouvrir et fermer un fichier :}
			
			\begin{lstlisting}
				int open(const char* pathname, int flags, mode_t mode);
			\end{lstlisting}
			\begin{itemize}
				\item \textbf{retour :} un "file descriptor" \lstinline!fd! >= 0, prend la plus petite valeur disponible (non ouvert) ou \textbf{-1} si l'ouverture échoue, auquel cas errno contient le code d'erreur
				\item \textbf{\lstinline!pathname! :} chemin du fichier à ouvrir
				\item \textbf{\lstinline!flags! :} mode d'accès, contient une ou plusieurs valeurs dont obligatoirement une parmi O\_RDONLY, O\_WRONLY ou O\_RDWR
				\item \textbf{\lstinline!mode! :} indique les permissions à utiliser si un nouveau fichier est créé, doit être fourni lorsque O\_CREAT est spécifié dans \lstinline!flags! sinon ignoré
			\end{itemize}
			
			\begin{lstlisting}
				int close(int fd);
			\end{lstlisting}
			\begin{itemize}
				\item \textbf{retour :} 0 ou -1 en cas d'échec
				\item \textbf{\lstinline!fd! :} descripteur du fichier à fermer
			\end{itemize}
			
		~\\\textbf{Lire et écrire dans un fichier :}\hspace*{-2em}
			\begin{lstlisting}
				ssize_t read(int fd, void* buf, size_t count);
			\end{lstlisting}
			\begin{itemize}
				\item \textbf{retour :} -1 en cas d'échec, et la position de la tête de lecture est indéfinie, sinon, renvoie le nombre d'octets lus (0 en fin de fichier), et avance la tête de lecture de ce nombre
				\item \textbf{\lstinline!fd! :} descripteur du fichier à lire
				\item \textbf{\lstinline!buf! :} pointe sur un tampon (buffer) où sont stockées les données lues 
				\item \textbf{\lstinline!count! :} nombre d'octets à lire, résultat indéfini si \lstinline!count! est supérieur à \lstinline!SSIZE_MAX!
			\end{itemize}
			
			\begin{lstlisting}
				ssize_t write(int fd, const void* buf, size_t count);
			\end{lstlisting}
			\begin{itemize}
				\item \textbf{retour :} -1 en cas d'échec, sinon, renvoie le nombre d'octets écrits et avance la tête de lecture de ce nombre
				\item \textbf{\lstinline!fd! :} descripteur du fichier dans lequel on va écrire
				\item \textbf{\lstinline!buf! :} pointe sur un tampon (buffer) où sont stockées les données à écrire 
				\item \textbf{\lstinline!count! :} nombre d'octets à écrire
			\end{itemize}
			
		~\\\textbf{Se déplacer dans un fichier :}\hspace*{-2em}
			\begin{lstlisting}
				off_t lseek(int fd, off_t offset, int whence);
			\end{lstlisting}
			\begin{itemize}
				\item \textbf{retour :} -1 en cas d'échec, sinon, renvoie le nouvel emplacement, mesuré en octets depuis le début du fichier
				\item \textbf{\lstinline!offset! :} nombre d'octets de déplacement 
				\item \textbf{\lstinline!whence! :} SEEK\_SET : la tête est placée à \lstinline!offset! octets depuis le début du fichier, SEEK\_CUR : la tête est avancée de \lstinline!offset! octets, SEEK\_END : la tête est placée à la fin du fichier plus \lstinline!offset! octets
			\end{itemize}
			
		\subsubsection*{Exemples :}
			On peut réécrire une version personnalisée des fonctions standard d'entrée / sortie en utilisant des appels-systèmes.
			
			\begin{lstlisting}
				int copie (char* fic1, char* fic2) 
				{
					int fd1 = open(fic1, O_RDONLY);
					int fd2 = open(fic2, O_WRONLY);
					
					int nb;
					char buf[1024];
					
					// tq nb != 0 et != -1
					while ( (nb = read(fd1, buf, 1024)) > 0 )
						if ( write(fd2, buf, nb) != nb )
							return -1;
							
					return 1;
				}
			\end{lstlisting}
		
			\begin{lstlisting}
				int myGetChar (int fd) 
				{
					int retour;
					unsigned char c;
					
					if ( read(fd, &c, 1) == 1 )
						retour = c;
					else
						retour = EOF;
						
					return retour;
				}
			\end{lstlisting}
			
						
			\begin{lstlisting}
				int myGetCharBuffer (int fd) 
				{
					// on effectue un read une fois que tous les MAX caractères sont lus, et on mémorise la tête de lecture pour lire un caractère  
					
					// variables static qui ne se réinitialisent pas à chaque appel de la fonction
					static unsigned char buf[MAX];
					static int nbCar = 0;	// nombre de caractères dans buf
					static char* p;	// tête de lecture
					
					int retour; 
					
					// on fait un read si le buffer est vide
					if ( nbCarac == 0 ) {
						nbCarac = read(fd, buf, MAX);
						p = buf;
					}
					
					// si le buffer contient des caractères, on en lit le premier
					if ( nbCarac > 0)
						retour = *p;	// valeur à la tête de lecture
					else 
						retour = EOF;	// plus rien à lire
						
					nbCarac--;
					p++;	// on avance la tête de lecture
						
					return retour;
				}
			\end{lstlisting}
		
		
		
		
		
		
		
		
	