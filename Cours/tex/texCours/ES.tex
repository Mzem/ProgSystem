\section{Entrées / Sorties}
	Les entrées / sorties traitent des données qu'on peut lire ou écrire à partir de fichiers. Il existe 2 niveaux de gestion des entrées / sorties et fichiers :
	\begin{itemize}
		\item Entrées / sorties effectuées immédiatement (avec des appels-systèmes)
		\item Entrées / sorties où les données sont mises en mémoire temporaire \textbf{\textit{buffer}} (avec des fonctions standard)
	\end{itemize}
	
	\subsection{Entrées / Sorties standard}
		Toutes les fonctions décrites ici sont déclarées dans la bibliothèque \lstinline!stdio.h!.
		
		\noindent\paragraph{}\textbf{Ouvrir et fermer un fichier :}
			
			\begin{lstlisting}
				FILE* fopen(const char* nomDuFichier, const char* modeOuverture);
				// renvoie NULL si l'ouverture échoue
				
				int fclose(FILE* pointeurSurFichier);
				// renvoie 0 si la fermeture réussit, EOF sinon
			\end{lstlisting}
		
		\noindent\paragraph{}\textbf{Écrire dans un fichier :}
		
			\begin{lstlisting}
				int fputc(int caractere, FILE* pointeurSurFichier);
				// écrit un seul caractère à la fois dans le fichier
				
				char* fputs(const char* chaine, FILE* pointeurSurFichier);
				// écrit une chaîne dans le fichier
				
				int fprintf(FILE* pointeurDeFichier, const char *format, ...); 
				// écrit une chaîne formatée dans le fichier, fonctionnement quasi-identique à printf
				
				// retournent EOF si écriture échoue
			\end{lstlisting}
			
		\noindent\paragraph{}\textbf{Lire à partir d'un fichier :}
		
			\begin{lstlisting}
				int fgetc(FILE* pointeurDeFichier);
				// lit un caractère, et avance la tête de lecture
				// retourne le caractère lu ou EOF sinon
				
				char* fgets(char* chaine, int nbreDeCaracteresALire, FILE* pointeurSurFichier);
				// lit au maximum une ligne et s'arrête au premier \lstinline!\n!
				// <nbreDeCaracteresALire> : pour s'arrêter le lire avant la fin de ligne, sert à définir une taille max
				// retourne NULL si elle ne peut rien lire
				
				int fscanf(FILE* pointeurDeFichier, const char *format, ...);
				// écrit une chaîne formatée
			\end{lstlisting}
			
		\noindent\paragraph{}\textbf{Se déplacer dans un fichier :} la tête de lecture est une sorte de curseur virtuel dans un fichier qui indique la position de lecture / écriture actuelle.
		
			\begin{lstlisting}
				long ftell(FILE* pointeurSurFichier);
				// indique la position actuelle dans le fichier
				
				int fseek(FILE* pointeurSurFichier, long deplacement, int origine);
				// déplace le curseur de <deplacement> caractères à partir de la position <origine>
				// <deplacement> peut être positif (en avant), 0 ou négatif (en arrière)
				// <origine> peut prendre les constantes SEEK_SET (début), SEEK_CUR (actuelle) ou SEEK_END (fin)
				
				void rewind(FILE* pointeurSurFichier);
				// retour au début
			\end{lstlisting}
			
		
	\subsection{Entrées / Sorties système}
		Les fonctions d'entrée / sortie système sont des appels-systèmes communiquant directement avec le noyeau. Les fonctions d'entrée / sortie standard font eux-même appel à ces fonctions système.
		
		
	